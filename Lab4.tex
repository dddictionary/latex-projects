%++++++++++++++++++++++++++++++++++++++++
% Don't modify this section unless you know what you're doing!
\documentclass[letterpaper,11pt]{article}
\usepackage{natbib}
\bibliographystyle{unsrtnat}
\usepackage{tabularx} % extra features for tabular environment
\usepackage{amsmath}  % improve math presentation
\usepackage{graphicx} % takes care of graphic including machinery
\usepackage[margin=1in,letterpaper]{geometry} % decreases margins
%\usepackage{cite} % takes care of citations
\usepackage[final]{hyperref} % adds hyper links inside the generated pdf file
\hypersetup{
	colorlinks=true,       % false: boxed links; true: colored links
	linkcolor=blue,        % color of internal links
	citecolor=blue,        % color of links to bibliography
	filecolor=magenta,     % color of file links
	urlcolor=blue         
}
%+++++++++++++++++++++++++++++++++++++++
\begin{document}

\title{Calculus Based Physics \\\textbf{Title}}
\author{Abrar Habib et. al.}
\date{December 22, 2021}
\maketitle

\begin{abstract}
TODO
\end{abstract}

\section{Introduction}

(Sample only, note the functions in bold-facing, italics, etc.) \\In 1820, \textbf{Biot and Savart} conducted an experiment in ... [1]. 

The \textit{Biot-Savart Law} is useful in ... [2].

A \textcolor{blue}{Helmholtz coil} is a device for ... and was named after ... These coils were widely used in ... to produce \underline{uniform magnetic fields} ...

The objective in this present lab is to ...

\section{Theory}

State, derive, and describe the important equations that you will 
need to use to compare theory and experiment. Include diagrams as 
necessary to help with visualizing variables. Leave mathematical d
etails of your derivations on the Appendix section. Below is an 
example to insert a numbered equation \ref{eq1} below

\begin{equation} \label{eq1} % the label is used to reference the equation
V=\frac{8\phi\Delta\pi a^{-5}}{\sqrt{3}\lambda\alpha\cdot\delta X \cdot\Sigma}+\nabla\vec{B}+\frac{\vec{E}}{\vec{v}}+\int \psi dL
\end{equation}

where $\psi$ is the distance to the Sun in units of km, $\lambda$ is something ... Always explain each variable once introduced. Do not introduce again at a later paragraph. 

Example on how to insert an equation on a separate line, unnumbered:
$$s_f=s_0+v_0t+\frac{1}{2}at^2$$
or you can state equations or variables within the paragraph like this $v_f^2=v_i^2+2a\Delta s$ or variable $\xi$.

\section{Methods}
First, we set up the system. We cut a string of length [INSERT LENGTH] 
and tied it to both our cart ($m_1$) and a spring. This spring was then 
attached to a 100 g mass ($m_2$). After assembling the system, we set up
the ramp. We propped one end of the ramp atop of two textbooks, creating 
an inclined plane, and tuned the heights of the track's legs to adjust 
the angle ($\Theta$) of the incline so the cart would remain stationary 
when placed on it, achieving equilibrium. To finish setting up, we placed 
two motion sensors to record the movements of the cart and the mass on 
DataStudio, leaving us with a set-up resembling Fig. 1:
\begin{figure}[htbp]
    \centerline{\includegraphics[scale=.3]{lab4 art.png}}
    \caption{The complete set-up of the system and inclined ramp, artistic liberties taken}
\end{figure}

Once everything was set up, we began our lab by monitoring the motion
of the system. We set the cart into motion with a push and recorded
roughly five seconds worth of data. Once we got a good graph of the
motion on DataStudio, we chose eight moments to 
analyze: t1 = 0 s, t2 = 0.24 s, t3 = 0.31 s, t4 = 0.45 s, t5 = 0.57 s, 
t6 = 0.95 s, t7 = 1.68 s, and t8 = 2.6 s. Using the information from 
these eight times, [and other stuff used in calculations], we were 
able to determine quantities like x, y1, y2, Us, Total Ug, Total T, 
and Total E.
To determine the value of x, or the length of the spring, 
we used the below equation:
\begin{equation}
    110.7-(m_1+m_2)
\end{equation}
where $m-1$ and $m_2$ are the length the cart has traveled and the length 
the mass has travelled respectively.
\\\\
To determine the value of y1 and y2, or [what are they lol] we used the 
below equation:
\begin{equation}
    x=y^2
\end{equation}
\\\\
To determine the value of Us, or spring potential 
energy, we used the below equation:
\begin{equation}
    x=y^2
\end{equation}
\\\\
To determine the total value of Ug, or total gravitational potential energy, we used the below equation:
\begin{equation}
    x=y^2
\end{equation}
\\\\
To determine the total value of T, or total kinetic energy, we used the below equation:
\begin{equation}
    x=y^2
\end{equation}
\\\\
To determine the total value of E, or total energy, we used the below equation:
\begin{equation}
    x=y^2
\end{equation}
\\\\
The value of Total E in particular is instrumental to our lab, as 
we are assessing the system's conservation of energy or lack 
thereof. After calculating the total at each of the eight 
moments, we then compared the values and determined if they 
were sufficiently similar enough to conclude that the energy 
of the system was conserved.



\section{Results and Analysis}

Describe all your results after presenting them. Include tabulated data set, larger tables can also be presented on the Appendix. Here's an example to insert a table -  Table~\ref{table1} is below:

\begin{table}[ht]
\begin{center}
\caption{Every table needs a caption. Note that the table caption is on top of the table! Note the consistency of precision of table values; do not forget the errors, labels, variables, and units.}
\label{table1} 
\begin{tabular}{ccc} %change to cc for 2 columns
\hline
\multicolumn{1}{c}{Distance, $d$ (km) } & \multicolumn{1}{c}{Voltage, $V\ (\pm 0.05$ V)} & \multicolumn{1}{c}{Current, $I$\ (mA $\pm 5$\%)}\\
\hline
1.2 $\pm$ 0.2 &  0.30 & 20 \\
1.6 $\pm$ 0.4 &  0.21 & 30 \\
2.5 $\pm$ 0.1 &  0.18 & 40 \\
5.9 $\pm$ 0.2 &  0.13 & 50 \\
\hline
\end{tabular}
\end{center}
\end{table}

\textbf{Use a full page to present important plotted findings, \textcolor{red}{don't be shy!}}(See Appendix for e.g.). Your plots should have axes labels with units, error bars, legend, captions, etc.

You can also discuss the sources of errors in this section; include ways on how you may want to improve the experimental methods performed. 



\section{Conclusion}
This section should be brief, concise, but complete. Directly answer 
your objectives, state your findings with errors, and conclude 
whether or not you were successful. Briefly explain if not successful.

\begin{thebibliography}{99}

\bibitem[\protect\citeauthoryear{Author}{2010}]{Author2010}
Author, A.N and Another, A. N., 2010, MNRAS, 431, 28.

\end{thebibliography}

\appendix

\section*{Appendix: Velocity measurements}

Below is an example large table; include mathematical derivations here as well.
\begin{table}[ht]
\begin{center}
\caption{Every table needs a caption.}
\label{table2} 
\begin{tabular}{cc} 
\hline
\multicolumn{1}{c}{distance (m)} & \multicolumn{1}{c}{V (km s$^-1$)} \\
\hline
0.0044151 &   0.0030871 \\
0.0021633 &   0.0021343 \\
0.0003600 &   0.0018642 \\
0.0023831 &   0.0013287 \\
0.0044151 &   0.0030871 \\
0.0021633 &   0.0021343 \\
0.0003600 &   0.0018642 \\
0.0023831 &   0.0013287 \\
0.0044151 &   0.0030871 \\
\hline
\end{tabular}
\end{center}
\end{table}

\end{document}
